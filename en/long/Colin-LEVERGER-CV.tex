%% start of file `template.tex'.
%% Copyright 2006-2013 Xavier Danaux (xdanaux@gmail.com).
%
% This work may be distributed and/or modified under the
% conditions of the LaTeX Project Public License version 1.3c,
% available at http://www.latex-project.org/lppl 

\documentclass[11pt,a4paper,sans]{moderncv}        % possible options include font size ('10pt', '11pt' and '12pt'), paper size ('a4paper', 'letterpaper', 'a5paper', 'legalpaper', 'executivepaper' and 'landscape') and font family ('sans' and 'roman')

\usepackage{verbatim} %comments

% moderncv themes
\moderncvstyle{classic}                             % style options are 'casual' (default), 'classic', 'oldstyle' and 'banking'
\moderncvcolor{green}                               % color options 'blue' (default), 'orange', 'green', 'red', 'purple', 'grey' and 'black'
%\renewcommand{\familydefault}{\sfdefault}         % to set the default font; use '\sfdefault' for the default sans serif font, '\rmdefault' for the default roman one, or any tex font name
%\nopagenumbers{}                                  % uncomment to suppress automatic page numbering for CVs longer than one page

\usepackage{pdfpages}

% character encoding
\usepackage[utf8]{inputenc}                       % if you are not using xelatex ou lualatex, replace by the encoding you are using

% adjust the page margins
\usepackage[scale=0.9, bottom=1.5cm,top=1cm]{geometry}
%\setlength{\hintscolumnwidth}{3cm}                % if you want to change the width of the column with the dates
%\setlength{\makecvtitlenamewidth}{10cm}           % for the 'classic' style, if you want to force the width allocated to your name and avoid line breaks. be careful though, the length is normally calculated to avoid any overlap with your personal info; use this at your own typographical risks...

% personal data
\name{Colin}{Leverger}
\title{Researcher, Data \& DevOps Engineer}                               % optional, remove / comment the line if not wanted
\address{City centre}{35000 RENNES}{FRANCE}% optional, remove / comment the line if not wanted; the "postcode city" and and "country" arguments can be omitted or provided empty
\phone[mobile]{+33(0)6~30~836~259}                   % optional, remove / comment the line if not wanted
%\phone[fixed]{+2~(345)~678~901}                    % optional, remove / comment the line if not wanted
%\phone[fax]{+3~(456)~789~012}                      % optional, remove / comment the line if not wanted
\email{colin.leverger@orange.fr}                               % optional, remove / comment the line if not wanted
\homepage{www.colinleverger.fr}                         % optional, remove / comment the line if not wanted
\extrainfo{\href{https://www.linkedin.com/in/colinleverger}{LinkedIn: lnkdin.me/cleverger} \\Last updated: May 2020}                 % optional, remove / comment the line if not wanted
%\photo[85pt][0pt]{img/colinleverger}                       % optional, remove / comment the line if not wanted; '64pt' is the height the picture must be resized to, 0.4pt is the thickness of the frame around it (put it to 0pt for no frame) and 'picture' is the name of the picture file
%\quote{Some quote}                                 % optional, remove / comment the line if not wanted

% to show numerical labels in the bibliography (default is to show no labels); only useful if you make citations in your resume
%\makeatletter
%\renewcommand*{\bibliographyitemlabel}{\@biblabel{\arabic{enumiv}}}
%\makeatother
%\renewcommand*{\bibliographyitemlabel}{[\arabic{enumiv}]}% CONSIDER REPLACING THE ABOVE BY THIS

% bibliography with mutiple entries
%\usepackage{multibib}
%\newcites{book,misc}{{Books},{Others}}
%----------------------------------------------------------------------------------
%            content
%----------------------------------------------------------------------------------
\begin{document}
%\begin{CJK*}{UTF8}{gbsn}                          % to typeset your resume in Chinese using CJK
%-----       resume       ---------------------------------------------------------
\makecvtitle

\section{Work experience}

\cventry
	{2017 - 2020}
	{Ph.D. candidate}
	{INRIA/IRISA, Orange Labs and Services}
	{Rennes - FRANCE}
	{}
	{
	Research in the field of Capacity Planning, time series, machine learning, Data Science. 
	\begin{itemize}
    \item Work both at INRIA and Orange Labs, respectively under the supervision of Thomas Guyet and Vincent Lemaire. 
    \item Development of Machine Learning algorithms using Python, Pandas, matplotlib, scipy, Keras \& GluonTS, Jupyter, R, and mining of time series datasets to make forecasts and predictions in the context of infrastructure performances. 
    \item Published 3 papers in national and international conferences (see below).
    \end{itemize}		
}

\cventry
	{2019}
	{Visiting researcher}
	{National Institute of Informatics}
	{Tokyo - JAPAN}
	{}
	{3 months research experience in NII labs at Tokyo.  
    %Work with Ryota KOBAYASHI in the field of time series and study of their seasonality.
    Developed a R Shiny dashboard for displaying massive amount of time series.
    Wrote one of my published paper.
    Initiate brand new research collaboration. 
    Laureate of two mobility grants: 'NII International Internship Program' and 'Bourse de mobilit\'e internationale 2019 - R\'egion Bretagne'.
    }


\cventry
	{2018}
	{Lab assistant}
	{E.N.S.A.I. Engineering School}
	{Bruz - FRANCE}
	{}
	{33 h as a lab assistant at E.N.S.A.I for master 2 students.
	Wrote the project subject, where students had to create a CLI application/web scraper in Python 3. Taught 4 groups of 5 students.}

\cventry
	{2017 - 2018}
	{Lab assistant}
	{Computer Science Department, University of Rennes}
	{Rennes - FRANCE}
	{}
	{52 h as a lab assistant in University of Rennes 1 for bachelor students. 40 h TP Scala, 14 h TP Web Programmation.}

\cventry
	{2014 - 2017}
	{Apprentice Software Engineer}
	{Orange Labs and Services}
	{Rennes - FRANCE}
	{}
	{Apprenticeship to develop software in Scala following a DevOps methodology in a team of 14 Metrology experts. 
	\begin{itemize}
	\item Implemented in Java and Scala a tool that crawled REST Maven repositories for creating a local knowledge base of metadata concerning Java libraries (last version, number of versions, etc.)
    \item Implemented in bash and Scala a software for infrastructure automatic testing. Managed the PHP, Apache and Nginx configuration files automatic discovery and retrieval for both online and offline analysis.
    \item Used Drools Java lib for creating expert rules for configuration management.
  	\item Participated in the testing of the whole software by designing and writing Scala tests with the ScalaTest library. Use of SBT for version and code management. Used Jenkins to integrate and deploy tests automatically.
    \item Developed a web interface for displaying continuously integrated and modified expertise rules. Use of various API. Deployed the service on cloud service OpenStack using Ansible automatic scripts.
    \item Developed a website using the asynchronous Scala Play framework.
    \item Used testing tools such as Gatling and jmeter to stress infrastructure, in order to maintain the service quality for Orange services.    
    \item Experimented with the ELK stack, Graphana and influxdb for massive log processing. Followed the performances of various servers and collected massive amount of data metrics.
    \item Developed some Machine Learning algorithms in the Capacity Planning speciality. Developed a PoC with R, Graphana and influxdb for technical and functional data display. Used Ansible and Jenkins for Continuous Integration of the various parts of the framework.
	\item Used Ansible, Scala, Spark, Elasticsearch, Logstash, Kibana, R, OpenStack, influxdb, etc. on a daily basis.	
	\end{itemize}
}

\cventry
    {2014}
    {Intern}
    {Athlone Institute of Technology}
    {Athlone - IRELAND}
    {}
    {3 months end-of-course internship; preparation of project subjects for students. Worked with Arduino, Java, HTML/CSS, PHP and C.}

\section{Higher education}

\cventry
	{2016 - 2017}
	{ERASMUS semester}
	{Roskilde University}
	{Roskilde - DENMARK}
	{}
	{6 months abroad, ERASMUS exchange in Denmark for the last academic semester of my Master Degree.\\ 
	\underline{Main subjects}: IT-architecture and user driven software design, Security, Big Data, Danish.}  % arguments 3 to 6 can be left empty


\cventry
	{2014 - 2017}
	{Master 2 Degree in Computer Science via an apprenticeship program at a highly selective graduate engineering school}
	{E.N.S.S.A.T.}
	{Lannion - FRANCE}
	{}
	{Curriculum Networking, Multimedia \& Computing Science.\\
	Participated to a lot of software development projects, some of those can be found in my web portfolio \href{https://colinleverger.fr}{https://colinleverger.fr}\\
    \underline{Main subjects}: Network Applications, Systems Architecture, Design of Services \& of User Interfaces.}  % arguments 3 to 6 can be left empty

\cventry
	{2012 - 2014}
    {Two year Undergraduate University Degree of Technology in Electronics and Computing}
    {Université de Rennes 1}
    {Rennes - FRANCE}
    {}
    {\underline{Main subjects}: Programming, Microchip, Telecommunication, Digital and Analogic Electronics.}

\section{Publications}

\cventry
    {Nov. 2019}
    {IDEAL 2019 - Main track paper}
    {English}
    {Manchester - UNITED KINGDOM}
    {}
    {Leverger, C., Malinowski, S., Guyet, T., Lemaire, V., Bondu, A., Termier, A. "Toward a framework for seasonal time series forecasting using clustering.", in IDEAL2019.}

\cventry
    {Sep. 2018}
    {ECML 2018 - Workshop paper}
    {English}
    {Dublin - IRELAND}
    {}
    {C. Leverger, V. Lemaire, S. Malinowski, T. Guyet and L. Rozé, "Day-ahead time series forecasting : application to capacity planning.", in AALTD workshop ECML 2018.}

\cventry
    {Jan. 2018}
    {EGC 2018 - Demo paper}
    {French}
    {Paris - FRANCE}
    {}
    {C. Leverger, R. Marguerie, V. Lemaire, T. Guyet, and S. Malinowski, "PerForecast : un outil de prévision de l'évolution de séries temporelles pour le planning capacitaire.", in EGC 2018, vol. RNTI-E-34, pp.455-458.}

\section{Personal projects}

\cventry
	{2014 - 2018}
    {Coursera MOOCs}
    {Coursera}
    {Internet}
    {}
    {Followed and completed more than 10 MOOCs, including ‘Functional Programming Principles in Scala’, ‘Big Data Analysis with Scala and Spark’, ‘Introduction to Neurohacking with R’, etc. Certificates available upon request.}
    
\cventry
	{2014 - 2018}
    {Systems and Network Engineer}
    {On my own}
    {Rennes - FRANCE}
    {}
    {Management of my dedicated server. 
    Use of Docker to manage virtual services, Jenkins to do some DevOps and deploy automatically, Ansible to script and configure the server. Play with ELK stack and Graphana dashboards for monitoring the services and cyberattacks.
    }
    
\section{Skills}
\subsection{Language}
\cvitem{French}{Native}{}
\cvitem{English}{Level C1+}{}
\cvitem{Danish}{Notions}{}
\cvitem{Japanese}{Notions}{}

\subsection{Computing}
\cvitem{Languages}{Python, R, Scala, Java, C, HTML/CSS, PHP, JavaScript, Bash/scripting, SQL, NoSQL, \LaTeX}
\cvitem{Paradigms}{Functional, imperative, object-oriented, concurrent programming, real-time programming}
\cvitem{Tools}{Pandas, NumPy, scikit-learn, Keras/Tensorflow, Matplotlib, Jupyter Notebook, R, Shiny, ggplot2, Git, Spark, influxdb, massive grid computing, Khiops coclustering tool @Orange, Jenkins, Ansible, Docker, Spark, ELK, OpenStack, texmaker, IntelliJ, MySQL, mongoDB, xampp, SSH, Neo4j, Gephi, Jmeter, Gatling}

\subsection{Interests and Hobbies}
\cvitem{Music}{Creation and composition, beatmaking, recording and sound engineering for my own band. We interpret only our original creations, that are available online on streaming services. Intensive use of Ableton 10 and of machines, both for music production and live performances.}
\cvitem{Sport}{Practice Ashtanga Yoga once a week at an intermediate level. Regular runner. Occasional swimmer.}
\cvitem{Kaggle}{Improving my data science skills on competitions and challenges on the Kaggle platform.}
\cvitem{Code}{Several projects, from System Administration on my dedicated server to contributions on Open Source projects (see \href{https://github.com/ColinLeverger}{https://github.com/ColinLeverger} for details.)}

\section{References}
\cventry{Manager}{Régis Marguerie}{\href{mailto:regis.marguerie@orange.com}{regis.marguerie@orange.com}}{+33 (0)6 07 585 086}{manager at Orange}{}
\cventry{Supervisor}{Vincent Lemaire}{\href{mailto:vincent.lemaire@orange.com}{vincent.lemaire@orange.com}}{+33 (0)2 96 073 107 }{Ph.D. supervisor at Orange}{}
\cventry{Supervisor}{Thomas Guyet}{\href{mailto:thomas.guyet@irisa.fr}{thomas.guyet@irisa.fr}}{+33 (0)2 99 847 594}{Ph.D. supervisor at INRIA}{}

\end{document}

%% end of file `template.tex'.

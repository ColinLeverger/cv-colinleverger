%!TEX TS-program = xelatex
\documentclass[]{friggeri-cv}
\usepackage{afterpage}
\usepackage{hyperref}
\usepackage{ucs}
\usepackage[utf8x]{inputenc}
\usepackage{color}
\usepackage{xcolor}
\hypersetup{
 pdftitle={Colin LEVERGER},
 pdfauthor={Colin LEVERGER},
 pdfsubject={CV},
 pdfkeywords={CV, Colin LEVERGER, Ph.D. en ML | Leader IA \& Chercheur @CyberSOC},
 colorlinks=false, % aucune couleur de bordure de lien
 allbordercolors=white % couleur de bordure blanche pour tous
}
% \addbibresource{bibliography.bib}
\RequirePackage{xcolor}
\definecolor{pblue}{HTML}{4CAF50}

\begin{document}
\header{Colin}{LEVERGER - Ph.D.}
{Ph.D. en ML | Leader IA \& Chercheur @CyberSOC}

% Texte factice pour ajouter un séparateur 
\fcolorbox{white}{gray}{\parbox{\dimexpr\textwidth-2\fboxsep-2\fboxrule}{%
 .....
 }}

% Dans l'encadré, chaque nouvelle ligne force un saut de ligne
\begin{aside}
 \includegraphics[scale=0.01]{img/white.png}
 \section{Adresse}
 Centre ville
 35000 RENNES
 FRANCE
 ~
 \section{Tél}
 +33 (0)6 33 50 47 22
 ~
 \section{E-mail}
 \href{mailto:colin.leverger@orange.fr}{\textbf{colin.leverger@}\\orange.fr}
 \section{Web \& Git}
 \href{http://www.colinleverger.fr}{colinleverger.fr}
 \href{https://www.linkedin.com/in/colinleverger}{lnkdin.me/cleverger}
 \href{https://github.com/ColinLeverger}{github/ColinLeverger}
 ~
 \section{Tech.}
 \textbf{Principal}:
 \emph{Python}: 7+ ans
 \emph{Bash/Unix}: 10+ ans
 \textbf{Expérimenté}:
 \emph{Scala/Java}: 4 ans
 \emph{HTML/PHP}: 4 ans
 \emph{R}: 3 ans
 \textbf{Utilisés par le passé}:
 \emph{JavaScript}: 2 ans
 \emph{SQL}: 1 an
 ~
 \section{Langues}
 \textbf{Français}: natif
 \textbf{Anglais}: C1+
 ~
 \section{DevOps}
 \emph{Docker}: 10+ ans
 \emph{CI/CD}: 10+ ans
 \emph{Gitlab-CI}: 5+ ans
 \emph{GCP/k8s}: 2+ ans
 ~
 \section{Intérêts}
 Data Science,
 LLMs,
 Cybersécurité,
 Intégration continue,
 Big Data, etc.
 ~
 \section{Soft Skills}
 Pragmatisme
 Méthode
 Communication
 Autonomie
 Curiosité
 Créativité
 Ténacité
 % Passion
 %\hspace*{-1.3in}
 %\includegraphics[scale=0.55]{img/personal.png}
 ~
\end{aside}
\emph{Leader IA pragmatique et orienté valeur, axé sur les stratégies d'architecture et la conduite du changement pour l'adoption de l'IA en cybersécurité et en opérations. Chercheur et directeur de thèse de doctorat dans le domaine d'AgenticAI.}
\vspace{-0.5em}
\section{Expérience}
\vspace{-0.5em}
\begin{entrylist}
 \entry
 {01/24 - Now}
 {Lead IA \& Data Cyber | Orange Graduate Program}
 {\href{http://www.orange.com/en/home}{Orange France}, Rennes - FRANCE}
 {Architecture, réalisation technique et pilotage pour créer un socle Données/IA pour le SOC d’Orange France. Conduite du changement en environnement opérationnel exigeant; focus sur la valeur (apprentissage automatique, LLMs); coordination métier/opérations; mentorat de stagiaires et encadrement d’un doctorat.}
 \entry
 {09/23 - 12/23}
 {Consultant IA \& LLMs | Orange Graduate Program}
 {\href{https://www.curebot.fr/}{EspritsCollaboratifs}, Paris - FRANCE}
 {Déploiement et évaluation de LLMs (LLaMA, Hugging Face, Azure OpenAI). Protocoles expérimentaux; environnements Cloud (OVH, Scaleway); analyses qualitatives/quantitatives et visualisations; recommandations sur sécurité, coûts et éthique. 4 mois en autonomie.}
 \entry
 {01/21 - 08/23}
 {Ingénieur données | Orange Graduate Program}
 {\href{http://www.orange.com/en/home}{Orange Recherche}, Rennes - FRANCE}
 {Ingénierie des données au sein d’une équipe expérimentée. Développement en Python de solutions Cloud (GCP) pour les Data Scientists et développeurs. Sujets: cybersécurité, outillage, gestion de projet, relation fournisseurs, recherche en science des données. Encadrement de stagiaires.}
 \entry
 {02/19 - 05/19}
 {Chercheur invité}
 {National Institute of Technology, Tokyo - JAPON}
 {3 mois de recherche dans les laboratoires du NII à Tokyo. } % Lancement d’une nouvelle collaboration de recherche.}
 \entry
 {10/17 - 11/20}
 {Doctorant}
 {\href{http://www.orange.com/en/home}{Orange Recherche} \& \href{https://www.inria.fr/fr}{INRIA}, Rennes - FRANCE}
 {Recherche en IA prédictive appliquée au planning capacitaire. Développement d’algorithmes d’apprentissage automatique (Pandas, Python, R) et exploration de séries temporelles pour prédire les performances d’infrastructure. 3 articles publiés (détails et références sur demande).}
 \entry
 {2017 - 2025}
 {Chargé de TP, jury}
 {Université de Rennes 1 et E.N.S.A.I, Rennes - FRANCE}
 {$\sim$250 h de travaux pratiques à l’Université de Rennes 1 ($\sim$100 h) et à l’ENSAI ($\sim$150 h), niveaux licence et master. Propositions de projets Python 3 (outils en ligne de commande, extraction Web).}
 \entry
 {09/14 - 09/17}
 {Apprenti Ingénieur Logiciel}
 {\href{http://www.orange.com/en/home}{Orange Recherche}, Rennes - FRANCE}
 {Apprentissage pour développer des logiciels en Scala selon une méthodologie DevOps.% au sein d'une équipe de 14 experts en métrologie. %\underline{Projet final}: dév. de
 % Implémentation en bash et Scala d'un logiciel pour des tests automatiques d'infrastructure. 
 % Utilisation d'outils de test tels que Gatling et jmeter pour mettre l'infrastructure sous contrainte. 
 % Expérimentation de la chaîne ELK, Grafana et influxdb pour le traitement massif de logs.
 % Développement d'algorithmes de Machine Learning dans la spécialité Capacity Planning. 
 % Utilisation de Scala, Spark, Elasticsearch, Logstash, Kibana, R, Openstack, influxdb, Gatling, Jmeter, etc...
 }
 \vspace{-2.5em}
 % \entry
 % {02/15 - Now}
 % {Ingénieur systèmes et réseaux}
 % {Indépendant, Rennes - FRANCE}
 % {Gestion de mon serveur dédié. 
 % Utilisation de Docker pour gérer des services virtuels, Jenkins pour faire du DevOps et déployer automatiquement, Ansible pour scripter et configurer le serveur. Expérimentations avec la chaîne ELK et des tableaux de bord Grafana pour le monitoring des services et des cyberattaques.}
 % \entry
 % {04/14 - 07/14}
 % {Stage}
 % {\href{http://www.ait.ie/}{Athlone Institute of Technology}, Athlone - IRLANDE}
 % {Stage de fin d'études de trois mois ; préparation de sujets de projets pour les étudiants. Travail avec \emph{Arduino}, \emph{Java}, \emph{HTML/CSS}, \emph{PHP} et \emph{C}.}
\end{entrylist}

%\section{Recherche \& Publications}
%\begin{entrylist}
% \entry
% {10/19}
% {Conférence IDEAL (C)}
% {Manchester - ROYAUME-UNI}
% {"Toward a framework for seasonal time series forecasting using clustering.", Leverger, C., Malinowski, S., Guyet, T., Lemaire, V., Bondu, A., \& Termier, A.}
% \entry
% {09/18}
% {Atelier AALTD à ECML (A)}
% {Dublin - IRLANDE}
% {"Day-ahead time series forecasting : application to capacity planning.", Leverger, C., Lemaire, V., Malinowski, S., Guyet, T., \& Rozé, L.}
%\end{entrylist}

\section{Enseignement supérieur}
\vspace{-0.5em}

\begin{entrylist}
 \entry
 {09/16 - 02/17}
 {ERASMUS}
 {\href{http://www.ruc.dk/en/}{Roskilde University}, Roskilde - DANEMARK}
 {5 mois à l'étranger, échange ERASMUS au Danemark pour le dernier semestre académique de mon Master. \underline{Matières principales}: architecture des SI et conception logicielle centrée utilisateur, sécurité, Big Data, danois.}
 \entry
 {09/14 - 09/17}
 {Master 2 en Ingénierie Informatique}
 {\href{http://www.enssat.fr}{E.N.S.S.A.T.}, Lannion - FRANCE}
 {Parcours Réseaux, Multimédia \& Informatique. \underline{Matières principales}: Applications réseaux, Architecture des systèmes, Conception de services \& d'interfaces utilisateurs. } %Participation à de nombreux projets de développement logiciel, certains sont disponibles dans mon portfolio web \href{https://colinleverger.fr}{https://colinleverger.fr}}
 \vspace{-2.5em}

 %\entry
 % {2012 - 2014}
 % {Diplôme universitaire de technologie de deux ans en Électronique et Informatique}
 % {\href{https://iut-rennes.univ-rennes1.fr/les-6-departements/genie-electrique-informatique-industrielle}{Université de Rennes 1}, Rennes - FRANCE}
 % {Matières principales : Programmation, Microcontrôleur, Électronique numérique et analogique.}
 % \vspace{-1.2em} 
 % \entry
 % {2016 - 2017}
 % {MOOC et cours en ligne}
 % {Coursera et edX, e-learning}
 % {Apprentissage du paradigme fonctionnel avec le langage Scala. Le cours est complété par une série de projets de programmation comme devoirs. Apprentissage de Google Cloud Platform : SDKs \& CLI, gestion des VMs, Scalabilité, ...}
\end{entrylist}

%\section{Certifications}
%\begin{entrylist}
% \entry
% {07/16 - 02/17}
% {Spécialisation Scala}
% {Coursera, E-learning}
% {Apprentissage du paradigme fonctionnel avec le langage Scala. Le cours est complété par une série de projets de programmation comme devoirs.}
%\entry
% {01/17 - 02/17}
% {Spécialisation Google Cloud Platform}
% {Coursera, E-learning}
% {Apprentissage de Google Cloud Platform : SDKs \& CLI, gestion des VMs, Scalabilité, ...}
%\end{entrylist}

\section{Compétences}
\vspace{-0.5em}

\begin{entrylist}
 % \entry
 % {Informatique}
 % {Langages}
 % {}
 % {Python, R, Scala, Java, C, HTML/CSS, PHP, JavaScript, Bash/scripting, SQL, NoSQL, \LaTeX}
 \entry
 % {}
 {Informatique}
 % {Paradigmes}
 % {}
 % {Programmation fonctionnelle, impérative, orientée objet, concurrente, temps réel}
 % \entry
 % {}
 {Outils notables}
 {}
 {FastAPI, Gitlab-CI, Pandas, NumPy, scikit-learn, Tensorflow, Matplotlib, Jupyter Notebook, R, ggplot2, Git, Spark, influxDB, calcul distribué massif, outil de coclustering Khiops @Orange}
 % \entry
 % {Loisirs}
 % {Musique}
 % {}
 % {Création et composition, beatmaking, enregistrement et ingénierie du son pour mon propre groupe. Nous interprétons uniquement nos créations originales, disponibles en ligne sur les plateformes de streaming. Utilisation intensive d'Ableton 10 et de machines, à la fois pour la production musicale et les performances live.}
 \entry
 {}
 {MOOC Coursera \& Certifications}
 {}
 {SANS FOR508 Forensics 2025, Certification GCP Cloud Architect 2023, Spécialisation Scala, Data Engineering GCP, etc.}
 \entry
 {Loisirs}
 {Sport}
 {}
 {Ashtanga Yoga, niveau intermédiaire. Coureur, grimpeur et nageur régulier.}
 \vspace{-2.5em}

\end{entrylist}
\end{document}
